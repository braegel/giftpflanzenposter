\documentclass[a1,portrait]{a0poster}
\usepackage[utf8]{inputenc}
\usepackage{multicol}
\usepackage[T1]{fontenc}
\usepackage{ngerman}
\usepackage{hyperref}
\usepackage{graphicx}
\usepackage{lmodern}
\renewcommand*\familydefault{\sfdefault} %serifenlos

%Vorlagen
\newcommand{\pflanze}[3]{
#1
\begin{center}
\includegraphics[width=0.95\linewidth]{bilder/#2}
\end{center}
Giftige Teile: #3
}


\begin{document}
\VERYHuge
\begin{center}
  Giftpflanzen
\end{center}
\normalsize

Haftungsausschluss

Lizenz

Quelle: \url{https://de.wikipedia.org/wiki/Liste_giftiger_Pflanzen}
\begin{multicols*}{5}

\pflanze{Alpenveilchen}{AlpenveilchenMessenien.ps}{Blätter, Knolle}
%https://commons.wikimedia.org/wiki/File:Alpenveilchen_Messenien.jpg#mediaviewer/File:Alpenveilchen_Messenien.jpg

\pflanze{Aronstab}{Kreta-Aronstab.ps}{alle Pflanzenteile}
%https://commons.wikimedia.org/wiki/File:Kreta-Aronstab.JPG#mediaviewer/File:Kreta-Aronstab.JPG

\pflanze{Bärenklau}{640px-Illustration_Heracleum_sphondylium0.ps}{Kontakt: alle Pflanzenteile}
%https://de.wikipedia.org/wiki/B%C3%A4renklau#mediaviewer/File:Illustration_Heracleum_sphondylium0.jpg
\end{multicols*}


\end{document}
